%%%%%%%%%%%%%%%%%%%%%% Tipo de documento %%%%%%%%%%%%%%%%%%%%%%
\documentclass[12pt,aspectratio=169]{beamer}
%%%%%%%%%%%%%%%%%%%%%%%%%%%%%%%%%%%%%%%%%%%%%%%%%%%%%%%%%%%%%%%

%%%%%%%%%%%%%%%%%% Configuración de paquetes %%%%%%%%%%%%%%%%%%
\usepackage[T1]{fontenc}
\usepackage[utf8]{inputenc}
\usepackage[spanish,es-tabla]{babel}
\usepackage{amsmath}
\usepackage{amssymb}
\usepackage{amsfonts}
\usepackage{latexsym}
\usepackage{lipsum}
\usepackage{graphicx}
\usepackage{hyperref}
\usepackage{float}
\usepackage{url}
\usepackage{xcolor-material}
\usepackage[type={CC}, modifier={by-nc}, version={4.0}]{doclicense}
%%%%%%%%%%%%%%%%%%%%%%%%%%%%%%%%%%%%%%%%%%%%%%%%%%%%%%%%%%%%%%%

%%%%%%%%%%%%%%%%%% Información del documento %%%%%%%%%%%%%%%%%%
\title[Proyecto final]{Proyecto final de Ingeniería Mecatrónica}
\subtitle{Sistema de control para planta de tratamiento de efluentes}
\author[Gianluca \& Gabriel]{Gianluca Lovatto y Gabriel Aguirre}
\date{\today}
\titlegraphic{\includegraphics[scale=0.18]{img/logo.png}}
\institute[UNER FCAL]{Facultad de Ciencias de la Alimentación}
%%%%%%%%%%%%%%%%%%%%%%%%%%%%%%%%%%%%%%%%%%%%%%%%%%%%%%%%%%%%%%%

%%%%%%%%%%%%%%% Configuración de diapositivas %%%%%%%%%%%%%%%%%
\usetheme{Madrid}
\usecolortheme{default}
\usefonttheme{default}
\useinnertheme{default}
\useoutertheme{default}

\setbeamercovered{invisible}
\setbeamertemplate{navigation symbols}{}

\setbeamercolor{background canvas}{bg=MaterialGrey50}
\setbeamercolor{palette primary}{fg=MaterialGrey50,bg=MaterialIndigo700}
\setbeamercolor{palette secondary}{fg=MaterialGrey50}
\setbeamercolor{palette tertiary}{fg=MaterialGrey50}
%%%%%%%%%%%%%%%%%%%%%%%%%%%%%%%%%%%%%%%%%%%%%%%%%%%%%%%%%%%%%%%

%%%%%%%%%%%%%%%%%%%%% Inicio del documento %%%%%%%%%%%%%%%%%%%%
\begin{document}

\begin{frame}
    \maketitle
\end{frame}

\begin{frame}{Filmina}
    \lipsum[1]
\end{frame}

\begin{frame}[plain]
    \lipsum[2]
\end{frame}

{
    \setbeamercolor{background canvas}{bg=MaterialIndigo700}

    \begin{frame}{Fondo}
        \begin{center}
            \textcolor{MaterialGrey50}
            {
                \lipsum[3]
            }
        \end{center}
    \end{frame}
}

{
    \setbeamercolor{background canvas}{bg=MaterialIndigo700}

    \begin{frame}[plain]
        \begin{center}
            \textcolor{MaterialGrey50}{\textbf
            {
                RESALTADO!!
            }}
        \end{center}
    \end{frame}
}

% Referencias
\begin{frame}{Referencias}
    \begin{thebibliography}{10}
        \beamertemplatearticlebibitems
        \bibitem{Author1990}
            Kung Ching Chang
            \newblock{\em Infinite Dimensional Morse Theory and Multiple Solution Problems}.
            \newblock{Vol. 6. Progres in nonlinear Diferential Equation and Their Aplications. Boston, Birkhauser, 1991.}
        
        \beamertemplatebookbibitems
        \bibitem{Author19901}
            Kanischka Perera.
            \newblock{\em Nontrivial groups in p-Laplacian problems via the Yang index}.
            \newblock{The \LaTeX\ Companion. In Topol. Methodos Nolinear Anal. 21.2 (2003) , pp 301-303}
        
        \beamertemplateonlinebibitems
        \bibitem{Author2019}
            Aprendiendo \LaTeX
            \newblock{\em Página de Facebook}.
            \newblock{Manuel Merino}
    \end{thebibliography}
\end{frame}

\begin{frame}{Licencia}
    \doclicenseThis
\end{frame}

\end{document}
%%%%%%%%%%%%%%%%%%%%%%%%%%%%%%%%%%%%%%%%%%%%%%%%%%%%%%%%%%%%%%%
